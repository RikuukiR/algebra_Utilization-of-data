\documentclass[12pt,a4paper]{jsarticle}
\usepackage[margin=20truemm]{geometry}

% 節番号を「第1節」形式に
\renewcommand{\thesection}{第\arabic{section}節}

\usepackage{amsmath}
\usepackage{ascmac}
\usepackage{amsthm}
\usepackage{amsfonts}
\usepackage{latexsym}
\usepackage{bm}
\usepackage{mathtools}
\usepackage{empheq}
\usepackage{amssymb}
\usepackage[dvipdfmx]{graphicx}
\usepackage{tikz}
\usepackage{textcomp}
\usepackage{enumitem}
\usepackage[most]{tcolorbox}
\usepackage[dvipdfmx]{xcolor}
\usepackage{environ}
\usepackage{etoolbox}
\usepackage{needspace}
\usetikzlibrary{intersections,calc,arrows.meta}

% 行間を少し広げる
\linespread{1.15}

% 分数を常に大きく表示
\everymath{\displaystyle}

% 数式環境前後の空白を調整(ドキュメント開始時に適用)
\AtBeginDocument{%
  \setlength{\abovedisplayskip}{1pt}%
  \setlength{\belowdisplayskip}{6pt}%
  \setlength{\abovedisplayshortskip}{1pt}%
  \setlength{\belowdisplayshortskip}{3pt}%
}

% 分数コマンドを再定義(分子と分母の両端に0.5文字分の空白を追加)
% 元のコマンドを保存
\let\oldfrac\frac
\let\olddfrac\dfrac
% 再定義(\, は0.5文字分の空白)
\renewcommand{\frac}[2]{\oldfrac{\,#1\,}{\,#2\,}}
\renewcommand{\dfrac}[2]{\olddfrac{\,#1\,}{\,#2\,}}

% 平均記号のバーを長くするコマンド
\newcommand{\mean}[1]{\overline{#1}}

% 添字内で使う余事象の表記(印刷時も見やすいように少し太めのバー)
\newcommand{\comp}[1]{\bar{#1}}

% 分数の穴埋めコマンド(下線なし)
% 分数用コマンド(表専用 - 分子を空欄にしてレイアウト保持)
\newcommand{\fracblank}[2]{%
  \ifshowanswer
    \textcolor{blue}{\frac{\,#1\,}{\,#2\,}}%  解答版:青色
  \else
    \frac{\,\phantom{#1}\,}{\,#2\,}%  空欄版:分子を透明化, レイアウト保持
  \fi
}

% enumerate環境のデフォルト設定(問題環境等で使用)
\setlist[enumerate,1]{label=(\arabic*), leftmargin=*, labelsep=0.5em}
\setlist[enumerate,2]{label=(\alph*), leftmargin=*, labelsep=0.5em}
\setlist[enumerate,3]{label=(\roman*), leftmargin=*, labelsep=0.5em}

% itemize環境を数学の場合分け用((ⅰ), (ⅱ)形式)に設定
\newcounter{itemizecounter}
\makeatletter
\renewenvironment{itemize}{%
  \setcounter{itemizecounter}{0}%
  \begin{list}{}{%
    \setlength{\leftmargin}{2.5em}%
    \setlength{\labelsep}{0.5em}%
    \setlength{\itemindent}{0em}%
    \def\makelabel##1{%
      \stepcounter{itemizecounter}%
      \hss\llap{(\roman{itemizecounter})}%
    }%
  }%
}{%
  \end{list}%
}
\makeatother

\usepackage[normalem]{ulem} % 下線用
\usepackage{stackengine} % 追加

% 丸囲み文字コマンド
\newcommand{\ctext}[1]{\raise0.2ex\hbox{\textcircled{\scriptsize{#1}}}}

% 定理スタイルのカスタマイズ(最後のピリオドを削除)
\newtheoremstyle{myplain}
  {3pt}   % Space above
  {3pt}   % Space below
  {\itshape}  % Body font
  {}  % Indent amount
  {\bfseries} % Theorem head font
  {}  % Punctuation after theorem head (空にしてピリオドを削除)
  {.5em}  % Space after theorem head
  {}  % Theorem head spec

\newtheoremstyle{mydefinition}
  {3pt}   % Space above
  {3pt}   % Space below
  {}  % Body font
  {}  % Indent amount
  {\bfseries} % Theorem head font
  {}  % Punctuation after theorem head (空にしてピリオドを削除)
  {.5em}  % Space after theorem head
  {}  % Theorem head spec

\theoremstyle{mydefinition}
\newtheorem{definition}{Definition}
\newtheorem*{definition*}{Definition}
\theoremstyle{myplain}
\newtheorem{theorem}{Theorem}
\newtheorem*{theorem*}{Theorem}

% 証明環境を再定義(証明環境内の最後に5行の空行を追加、□の前、ページを跨がないように)
\makeatletter
\renewenvironment{proof}[1][\proofname]{\par
  \needspace{8\baselineskip}%  証明環境全体が1ページに収まるように必要なスペースを確保
  \pushQED{\qed}%
  \normalfont \topsep6\p@\@plus6\p@\relax
  \trivlist
  \item[\hskip\labelsep
        \itshape
    #1\@addpunct{.}]\ignorespaces
}{%
  \par\vspace*{5\baselineskip}%  証明環境内の解答欄確保のため5行の空行を追加(□の前)
  \popQED\endtrivlist\@endpefalse
}
\makeatother

% ========================================
% 問題環境の設定
% ========================================
% 問題番号カウンター(単純な連番: 1, 2, 3, ...)
\newcounter{problemcounter}
\renewcommand{\theproblemcounter}{\arabic{problemcounter}}

% 問題環境(枠なし, シンプル)
\newenvironment{problem}[1][\relax]{%
  \refstepcounter{problemcounter}%
  \vspace{2em}%
  \noindent%
  \textbf{問題\theproblemcounter}%
  \ifx#1\relax\else~\textbf{#1}\fi%
  \par\vspace{0.8em}%
  \noindent%
}{%
  \par\vspace{2em}%
}

% 問題環境(枠付き)
\newenvironment{problembox}[1][\relax]{%
  \refstepcounter{problemcounter}%
  \vspace{2em}%
  \begin{itembox}[l]{\textbf{問題\theproblemcounter}%
  \ifx#1\relax\else~\textbf{#1}\fi}%
}{%
  \end{itembox}%
  \vspace{1.5em}%
}

% 例題環境(枠付き)- 【】内の番号を使用
\newenvironment{example}[1][\relax]{%
  \vspace{2em}%
  \begin{itembox}[l]{\textbf{例題}%
  \ifx#1\relax\else~\textbf{#1}\fi}%
}{%
  \end{itembox}%
  \vspace{1.5em}%
}

% 練習問題環境(枠付き)- 【】内の番号を使用
\newenvironment{exercise}[1][\relax]{%
  \vspace{2em}%
  \begin{itembox}[l]{\textbf{練習問題}%
  \ifx#1\relax\else~\textbf{#1}\fi}%
}{%
  \end{itembox}%
  \vspace{1.5em}%
}

% 解答環境
\newenvironment{solution}{%
  \vspace{1em}%
  \noindent%
  \ifshowanswer%
    \textbf{【解答】}\\%
  \fi%
}{%
  \par\vspace{1em}%
}

% 問題専用の解答環境(常に空欄, 手書きスペース確保)
% 使い方: \begin{problemsolution}[高さ] ... \end{problemsolution}
% デフォルト高さは15cm
% ※最後の空白は手動で \vspace{5\baselineskip} を追加してください
% 空欄時: 解答内容を白色で出力し, 証明終了記号(□)を同じ位置に配置
\NewEnviron{problemsolution}[1][15cm]{%
  \begin{proof}\mbox{}\\%
  \ifshowanswer%
    {\color{blue}\BODY}%
  \else%
    {\color{white}\BODY}%
  \fi%
  \end{proof}%
}

% 別解環境(解答表示時のみ表示, ピリオドなし)
\NewEnviron{altproof}{%
  \ifshowanswer%
    \par\vspace{1em}%
    \noindent\textbf{【別解】}\par%
    {\color{blue}\BODY\par\hfill$\square$}%
    \vspace{1em}%
  \fi%
}

% ========================================
% 解答表示の切り替え設定
% ========================================
% 解答を表示したい場合:\showanswertrue
% 解答を隠したい場合:\showanswerfalse
\newif\ifshowanswer
% ここを \showanswertrue に変更すると解答が表示されます
\showanswertrue
% 穴埋めコマンドの修正版(upLaTeX対応)
\newcommand{\Blank}[2][3cm]{%
  ~%
  \ifshowanswer
    \stackon[1pt]{\uline{\makebox[#1][l]{\hphantom{あ}}}}{{\normalsize\sffamily #2}}%
  \else
    \uline{\makebox[#1][l]{\hphantom{あ}}}%
  \fi
  ~%
}

% 正解に合わせて自動で空欄を作る(em単位版)
% 使用法: \fitblank[答え]{幅} または \fitblank{幅}(答えなしの場合)
% 太文字版: \fitblankbf[答え]{幅}
% 下線の両端に1文字分の空白を追加
\newcommand{\fitblank}[2][\relax]{%
  \ifx#1\relax
    % 答えが指定されていない場合(従来の使用法)
    \hspace{1em}\underline{\hspace*{#2em}}\hspace{1em}%
  \else
    % 答えが指定されている場合
    \ifshowanswer
      \hspace{1em}\underline{\makebox[#2em][c]{#1}}\hspace{1em}%
    \else
      \hspace{1em}\underline{\hspace*{#2em}}\hspace{1em}%
    \fi
  \fi
}

% 太文字版の空欄コマンド(答えは青色で表示)
% 下線の両端に1文字分の空白を追加
\newcommand{\fitblankbf}[2][\relax]{%
  \ifx#1\relax
    % 答えが指定されていない場合(従来の使用法)
    \hspace{1em}\underline{\hspace*{#2em}}\hspace{1em}%
  \else
    % 答えが指定されている場合
    \ifshowanswer
      \hspace{1em}\underline{\makebox[#2em][c]{\textcolor{blue}{\textbf{#1}}}}\hspace{1em}%
    \else
      \hspace{1em}\underline{\hspace*{#2em}}\hspace{1em}%
    \fi
  \fi
}

% 数式と文字の両方を太字にするコマンド
\newcommand{\fitblankbold}[2][\relax]{%
  \ifx#1\relax
    % 答えが指定されていない場合(従来の使用法)
    \underline{\hspace*{#2em}}%
  \else
    % 答えが指定されている場合
    \ifshowanswer
      \underline{\makebox[#2em][c]{\boldmath\textbf{#1}}}%
    \else
      \underline{\hspace*{#2em}}%
    \fi
  \fi
}

% ========================================
% tcolorbox環境(背景パターンで区別)
% ========================================

% 定義環境(赤系:薄い赤背景 + 赤枠)- 自動番号付き
% 注: #1はラベル(オプション), #2はタイトル(必須)
\newtcolorbox{definitionbox}[2][]{
    enhanced,
    colback=red!5,
    colframe=red!70!black,
    boxrule=0.8pt,
    sharp corners,
    left=10pt, right=10pt, top=8pt, bottom=8pt,
    fonttitle=\bfseries,
    code={\refstepcounter{definition}\if\relax\detokenize{#1}\relax\else\label{#1}\fi}, % #1が空でない場合のみラベル設定
    title={Definition \thedefinition\if\relax\detokenize{#2}\relax\else\quad\textbf{#2}\fi}
}

% 定理環境(青系:薄い青背景 + 青枠)- 自動番号付き
% 注: #1はラベル(オプション), #2はタイトル(必須)
\newtcolorbox{theorembox}[2][]{
    enhanced,
    colback=blue!5,
    colframe=blue!70!black,
    boxrule=1pt,
    sharp corners,
    left=10pt, right=10pt, top=8pt, bottom=8pt,
    fonttitle=\bfseries,
    code={\refstepcounter{theorem}\if\relax\detokenize{#1}\relax\else\label{#1}\fi}, % #1が空でない場合のみラベル設定
    title={Theorem \thetheorem\if\relax\detokenize{#2}\relax\else\quad\textbf{#2}\fi}
}

% 問題環境(濃い灰色背景 + 太線)
\newtcolorbox{problembox2}[1][]{
    enhanced,
    colback=black!10,
    colframe=black,
    boxrule=1.2pt,
    sharp corners,
    left=10pt, right=10pt, top=8pt, bottom=8pt,
    fonttitle=\bfseries,
    title=#1
}

% ========================================
% テクニック環境(緑系:薄い緑背景 + 緑枠)
% ========================================
\newcounter{technique}
\newtcolorbox{techniquebox}[2][]{
    enhanced,
    colback=green!5,
    colframe=green!70!black,
    boxrule=0.8pt,
    sharp corners,
    left=10pt, right=10pt, top=8pt, bottom=8pt,
    fonttitle=\bfseries,
    code={\refstepcounter{technique}\if\relax\detokenize{#1}\relax\else\label{#1}\fi},
    title={Technique \thetechnique\if\relax\detokenize{#2}\relax\else\quad\textbf{#2}\fi}
}

% ========================================
% インラインテクニックコマンド(数式環境内でも使用可能)
% ========================================
% シンプル版(数式内推奨)
\newcommand{\tech}[1]{%
    \textcolor{green!60!black}{\textit{(#1)}}%
}

% 強調版(通常テキスト用)
\newcommand{\technique}[1]{%
    \textcolor{green!70!black}{\textbf{【テクニック】}} #1%
}

% 背景色付き版(目立たせたい場合)
\newcommand{\tip}[1]{%
    \colorbox{yellow!20}{\textcolor{orange!70!black}{\textbf{💡}} #1}%
}

% ========================================
% 手書き用空欄コマンド(下線なし, スペース確保)
% ========================================

% 解答ブロック全体の表示/非表示
% 使い方: \answerblock{解答内容}
\newcommand{\answerblock}[1]{%
  \ifshowanswer
    \textcolor{blue}{#1}%  解答を青色で表示
  \else
    % 空欄時は何も表示しない
  \fi
}

% ========================================
% answer用コマンド(レイアウト基準は答えの幅, 空欄時も幅を確保)
% ========================================

% インライン用(テキスト・数式両用)
% 使い方: \answertext{答え}
% 空欄時: 答えの幅 + 両端に3em(3文字分)の余白(手書き用)
\newcommand{\answertext}[1]{%
  \ifshowanswer
    \textcolor{blue}{#1}%  解答を青色で表示(下線なし)
  \else
    \uline{\hspace{3em}\phantom{#1}\hspace{3em}}%  空欄時は答えの幅 + 両端3em(手書き用)+ 黒の下線
  \fi
}

% 数式ブロック用(align環境など内部で使用)
% 使い方: \answermath{数式}
% 空欄時: 答えの幅 + 両端に4em(4文字分)の余白(手書き用)
\newcommand{\answermath}[1]{%
  \ifshowanswer
    {\color{blue}#1}%  数式を青色で表示(下線なし)
  \else
    \underline{\hspace{4em}\phantom{#1}\hspace{4em}}%  空欄時は答えの幅 + 両端4em(手書き用)+ 黒の下線
  \fi
}

% align環境全体を解答表示機能の対象にする環境
% 使い方: \begin{alignanswer*} ... \end{alignanswer*}
% 空欄時: 前後に空白を追加(手書き用)
\NewEnviron{alignanswer*}{%
  \ifshowanswer
    \begingroup
    \color{blue}%
    \begin{align*}
      \BODY
    \end{align*}%
    \endgroup
  \else
    \vspace{1em}%
    \begingroup
    \color{white}%
    \begin{align*}
      \BODY
    \end{align*}%
    \endgroup
    \vspace{1em}%
  \fi
}

% 表専用コマンド(空欄にしてレイアウト保持)
% 使い方: \answertable{答え}
% 空欄版:透明化してレイアウト保持
\newcommand{\answertable}[1]{%
  \ifshowanswer
    \textcolor{blue}{#1}%  解答版:青色で表示
  \else
    \phantom{#1}%  空欄版:透明化, レイアウト保持
  \fi
}

% ========================================
% blank用コマンド(解答時も余白あり版)
% ※通常は answer用コマンドを使用してください
% ========================================

% インライン用(テキスト・数式両用)
% 使い方: \blanktext{答え}
% 解答時・空欄時とも: 答えの幅 + 両端に0.5em空白
\newcommand{\blanktext}[1]{%
  \ifshowanswer
    \hspace{0.5em}\textcolor{blue}{#1}\hspace{0.5em}%
  \else
    \uline{\hspace{0.5em}\phantom{#1}\hspace{0.5em}}%  空欄時は黒の下線付き
  \fi
}

% 数式ブロック用(align環境など内部で使用)
% 使い方: \blankmath{数式}
% 解答時・空欄時とも: 答えの幅 + 左1.5em, 右2.5em空白
\newcommand{\blankmath}[1]{%
  \ifshowanswer
    \hspace{1.5em}{\color{blue}#1}\hspace{2.5em}%
  \else
    \hspace{1.5em}\phantom{#1}\hspace{2.5em}%
  \fi
}

% 手書きスペース確保版(指定した高さの空白を作る)
% 使い方: \answerspace[高さ]{解答内容}
% 高さのデフォルトは5cm
\newcommand{\answerspace}[2][5cm]{%
  \ifshowanswer
    \textcolor{blue}{#2}%  解答を青色で表示
  \else
    \vspace{#1}%  指定した高さの空白を確保
  \fi
}

% 既存の青色空欄コマンド(下線付き版も追加)
% 青色版の fitblank
\newcommand{\fitblankblue}[2][\relax]{%
  \ifx#1\relax
    % 答えが指定されていない場合
    \underline{\hspace*{#2em}}%
  \else
    % 答えが指定されている場合
    \ifshowanswer
      \textcolor{blue}{\underline{\makebox[#2em][c]{#1}}}%
    \else
      \underline{\hspace*{#2em}}%
    \fi
  \fi
}

% 青色太文字版
\newcommand{\fitblankbfblue}[2][\relax]{%
  \ifx#1\relax
    \underline{\hspace*{#2em}}%
  \else
    \ifshowanswer
      \textcolor{blue}{\underline{\makebox[#2em][c]{\textbf{#1}}}}%
    \else
      \underline{\hspace*{#2em}}%
    \fi
  \fi
}

% 分数の青色版
% 分数用コマンド(表専用 - 分子を空欄にしてレイアウト保持)
\newcommand{\fracblankblue}[2]{%
  \ifshowanswer
    \textcolor{blue}{\frac{\,#1\,}{\,#2\,}}%  解答版:青色
  \else
    \frac{\,\phantom{#1}\,}{\,#2\,}%  空欄版:分子を透明化, レイアウト保持
  \fi
}


% subsection番号を「1.1」形式に(thesectionの影響を受けないように)
\renewcommand{\thesubsection}{\arabic{section}.\arabic{subsection}}

% カウンターの初期値を設定
\setcounter{section}{0}
\setcounter{subsection}{0}
\setcounter{definition}{0}
\setcounter{theorem}{0}

% ========================================
% カウンターの階層化設定(○.○.○形式)
% ========================================
% 定義・定理・表・図カウンターをsubsectionに従属させる
\counterwithin{definition}{subsection}
\counterwithin{theorem}{subsection}
\counterwithin{table}{subsection}
\counterwithin{figure}{subsection}

% 表示形式を「○.○.○」に変更
\renewcommand{\thedefinition}{\arabic{section}.\arabic{subsection}.\arabic{definition}}
\renewcommand{\thetheorem}{\arabic{section}.\arabic{subsection}.\arabic{theorem}}
\renewcommand{\thetable}{\arabic{section}.\arabic{subsection}.\arabic{table}}
\renewcommand{\thefigure}{\arabic{section}.\arabic{subsection}.\arabic{figure}}

\title{【中学数学】 データの活用}                % タイトル
\author{}                     % 著者(空白でOK)
\date{}                       % 日付(空白)

\begin{document}
\begin{center}
{\LARGE 【中学数学】 第5章 データの活用}

体系数学2/数研出版/代数編
\end{center}

% --- 著者と日付を右寄せで表示 ---
\vspace{5mm}
\hfill Riku Sugawara \\
\hfill 12.2025

% ========================================
% 各セクションの読み込み
% ========================================

% ========================================
% 第1節 データの整理
% ========================================

\section{データの整理}

% 度数分布表とヒストグラム
\subsection{度数分布表とヒストグラム}

\begin{definitionbox}[def:度数分布表]{\textbf{度数分布表}}
\begin{minipage}[t]{0.55\textwidth}
    データの範囲を適当に区切ったとき, 各区間に含まれるデータの個数を \textbf{度数} といい,

    各区間にその区間の度数を対応させて整理した右のような表を \textbf{度数分布表} という .

\end{minipage}%
\hfill
\begin{minipage}[t]{0.4\textwidth}
    \centering
    \vspace{0pt}

    \begin{table}[H]
    \centering
    \begin{tabular}{|r@{〜}l|c|}
    \hline
    \multicolumn{2}{|c|}{\textbf{階級(cm)}} & \textbf{度数(人)} \\
    \hline
    135{\scriptsize 以上} & 140{\scriptsize 未満} & 2 \\
    \hline
    140 & 145 & 4 \\
    \hline
    145 & 150 & 5 \\
    \hline
    150 & 155 & 8 \\
    \hline
    155 & 160 & 11 \\
    \hline
    160 & 165 & 9 \\
    \hline
    165 & 170 & 7 \\
    \hline
    170 & 175 & 4 \\
    \hline
    \multicolumn{2}{|c|}{\textbf{計}} & \textbf{50} \\
    \hline
    \end{tabular}
    \caption{身長の度数分布表}
    \label{tab:height_distribution}
    \end{table}
\end{minipage}
\end{definitionbox}

\begin{definitionbox}[def:階級]{\textbf{階級}}
    度数分布表において, 区切られた各区間を \textbf{階級} , 区間の幅を \textbf{階級の幅} ,

    各階級の中央の値を \textbf{階級値} という .
\end{definitionbox}

\begin{definitionbox}[def:ヒストグラム]{\textbf{ヒストグラム}}
    度数分布表を, 柱状のグラフで表したものを \textbf{ヒストグラム} という .

    \vspace{0.5em}
    ヒストグラムの各長方形の横の長さは階級の幅を表し, 高さは各階級の度数を表す .
\end{definitionbox}

\begin{definitionbox}[def:度数折れ線]{\textbf{度数折れ線}}
    ヒストグラムの各長方形の上の辺の中点を結んでできる折れ線グラフを \textbf{度数折れ線} という .

    \vspace{0.5em}
    ただし, 度数折れ線をつくる時は, ヒストグラムの左右の両端に度数が0の階級があるものと考える .
\end{definitionbox}
\newpage

% 追加例題1
\begin{example}[【教p.114 追加例題1】]
    表\ref{tab:height_distribution}において, 次の問いに答えなさい .
    \begin{enumerate}
        \item 階級の幅は何か .
        \item 階級の個数はいくつか .
        \item 階級140cm以上145cm未満の度数はいくつか .
        \item 階級140cm以上145cm未満の階級値はいくつか .
    \end{enumerate}
\end{example}

% 相対度数
\subsection{相対度数}

\begin{definitionbox}[def:相対度数]{\textbf{相対度数}}
    度数の合計に対する各階級の度数の割合を \textbf{相対度数} という .

    \vspace{0.5em}
    相対度数はふつう小数を用いて表す .
\end{definitionbox}

\begin{theorembox}[thm:相対度数]{\textbf{相対度数}}
    相対度数は次の式で求められる .

    \[
    \textbf{(相対度数)} = \, \dfrac{\textbf{(\answermath{\hspace{6em}})}}{\textbf{(\answermath{\hspace{6em}})}} \,
    \]

    \vspace{1em}
\end{theorembox}
\vspace{2em}

Definition\ref{def:相対度数}より, 相対度数の合計は必ず \hspace{3em} になる .

\vspace{1em}

また, 相対度数を使用するメリットは, 度数の合計が異なる複数の分布について,

\vspace{0.5em}

\hspace{3em} を比較することができる .


% 累積度数
\subsection{累積度数}

\begin{definitionbox}[def:累積度数]{\textbf{累積度数}}
    度数分布表において, 各階級以下または各階級以上の階級の度数をたし合わせたものを

    \textbf{累積度数} という .

    \vspace{0.5em}
    また, 累積度数を表にまとめたものを \textbf{累積度数分布表} という .
\end{definitionbox}

% 累積相対度数
\begin{definitionbox}[def:累積相対度数]{\textbf{累積相対度数}}
    累積度数についても, 度数の合計に対する各階級の累積度数の割合を考えることがある .

    \vspace{0.5em}
    この割合を \textbf{累積相対度数} という .
\end{definitionbox}

% 追加例題2
\begin{example}[【教p.119 追加例題2】]
    表\ref{tab:height_distribution}において, 次の問いに答えなさい .
    \begin{enumerate}
        \item 各階級の相対度数を求めなさい .
        \item 累積度数の列を加え, 累積度数分布表を作成しなさい .
        \item 累積相対度数を求めなさい .
        \item 身長が150cm以上の人の割合は何\%か求めなさい .
    \end{enumerate}
\end{example}

\begin{proof}
    
\end{proof}

\newpage

% ========================================
% 第2節 データの代表値
% ========================================

\section{データの代表値}

\begin{definitionbox}[def:代表値]{\textbf{代表値}}
    いくつかの値が集まったデータがあるとき、そのデータ全体の特徴を表す数値を、
    そのデータの \textbf{代表値} という。
\end{definitionbox}
\vspace{1em}

代表値には、平均値、中央値、最頻値がある。

\subsection{平均値}

\begin{definitionbox}[def:平均値]{\textbf{平均値}}
    $n$ 個の値が集まったデータがあるとする。

    これら $n$ 個の値の合計を個数 $n$ でわった値を、このデータの \textbf{平均値} という。
\end{definitionbox}

\begin{theorembox}[thm:平均値]{\textbf{平均値}}
    平均値は次の式で求められる。

    \[
    \textbf{(平均値)} = \dfrac{\textbf{(データの値の合計)}}{\textbf{(データの個数)}}
    \]
\end{theorembox}

% 練習5
\begin{exercise}[【教p.120 練習5】]
    次のデータは、ジョギングを日課にしているAさんが最近5日間に行ったジョギングの時間である。
    このデータの平均値を求めよ。

    \[
    23 \qquad\quad 18 \qquad\quad 35 \qquad\quad 27 \qquad\quad 42 \qquad \text{(単位は 分)}
    \]
\end{exercise}

    \begin{proof}\mbox{}\\
        \begin{alignanswer*}
            \textbf{(平均値)} &= \dfrac{23 + 18 + 35 + 27 + 42}{5} \\
            &= \dfrac{145}{5} \\
            &= 29 \text{(分)}
        \end{alignanswer*}
    \end{proof}
\newpage

\begin{definitionbox}[def:中央値]{\textbf{中央値}}
    \uwave{データを大きさの順に並べたとき}、その中央にくる値を、\textbf{中央値} または \textbf{メジアン} という。

    ただし、データの個数が偶数のとき、中央に2つの値が並ぶから、その2つの値の平均値を中央値とする。
\end{definitionbox}

% 練習6
\begin{exercise}[【教p.120 練習6】]
    次のデータは、あるクラスの生徒10人の英語のテストの得点である。
    このデータの中央値を求めよ。

    \[
    75 \qquad 38 \qquad 49 \qquad 88 \qquad 61 \qquad 83 \qquad 44 \qquad 67 \qquad 58 \qquad 95 \qquad \text{(単位は 点)}
    \]
\end{exercise}

    \begin{proof}\mbox{}\\
        10人の英語のテストの得点を、低い順に並べると

        \begin{alignanswer*}
            38 \qquad 44 \qquad 49 \qquad 58 \qquad 61 \qquad 67 \qquad 75 \qquad 83 \qquad 88 \qquad 95
        \end{alignanswer*}

        5番目と6番目の得点の平均値が中央値であるから

        \begin{alignanswer*}
            \textbf{(中央値)} &= \dfrac{61 + 67}{2} \\
            &= 64 \text{(点)}
        \end{alignanswer*}
    \end{proof}

\begin{definitionbox}[def:最頻値]{\textbf{最頻値}}
    データにおいて、最も個数の多い値を、そのデータの \textbf{最頻値} または \textbf{モード} という。

    \uwave{データが度数分布表に整理されているときは}、\textbf{度数が最も大きい階級の階級値を最頻値}とする。
\end{definitionbox}

% 練習7
\begin{exercise}[【教p.121 練習7改訂】]
    表\ref{tab:height_distribution}において、身長の最頻値を求めよ。
    このデータの最頻値を求めよ。
\end{exercise}
\newpage

度数分布表を利用したデータの平均値を求める方法を考えよう。データが度数分布表にまとめられているとき、\textbf{ある階級に含まれるデータは、すべてその階級の階級値をとるものと考えて} 平均値を求める。

\begin{theorembox}[thm:度数分布表を利用した平均値]{\textbf{度数分布表を利用した平均値}}
    度数分布表を利用した平均値は次の式で求められる。
    \[
    \textbf{(平均値)} = \dfrac{\{\textbf{\textbf{(階級値)}} \bm{\times} \textbf{\textbf{(度数)}}\} \textbf{の合計}}
    {\textbf{\textbf{(度数の合計)}}}
    \]
\end{theorembox}

% 追加例題3
\begin{example}[【教p.121 追加例題3】]
    表\ref{tab:height_distribution}において, 次の問いに答えなさい.
    \begin{enumerate}
        \item 身長の平均値を求めなさい.
        \item 身長の中央値を求めなさい.
    \end{enumerate}
\end{example}

\begin{proof}\mbox{}\\
    \begin{table}[H]
        \centering
        \begin{tabular}{|r@{〜}l|c|c|c|}
        \hline
        \multicolumn{2}{|c|}{\textbf{階級(cm)}} & \textbf{度数(人)} & \textbf{階級値} & \textbf{階級値}$\bm{\times}$\textbf{度数} \\
        \hline
        135{\scriptsize 以上} & 140{\scriptsize 未満} & 2 & \answertable{137.5} & \answertable{275} \\
        \hline
        140 & 145 & 4 & \answertable{142.5} & \answertable{570} \\
        \hline
        145 & 150 & 5 & \answertable{147.5} & \answertable{737.5} \\
        \hline
        150 & 155 & 8 & \answertable{152.5} & \answertable{1220} \\
        \hline
        155 & 160 & 11 & \answertable{157.5} & \answertable{1732.5} \\
        \hline
        160 & 165 & 9 & \answertable{162.5} & \answertable{1462.5} \\
        \hline
        165 & 170 & 7 & \answertable{167.5} & \answertable{1172.5} \\
        \hline
        170 & 175 & 4 & \answertable{172.5} & \answertable{690} \\
        \hline
        \multicolumn{2}{|c|}{\textbf{計}} & \textbf{50} & & \answertable{7860} \\
        \hline
        \end{tabular}
        \label{tab:height_average}
    \end{table}

    \begin{alignanswer*}
        \text{(平均値)} = \dfrac{7860}{50} = \answermath{157.2} \text{(cm)}\\
        \text{(中央値)} = \answermath{155.5} \text{(cm)}
    \end{alignanswer*}
    \vspace{-6em}
\end{proof}

\newpage

% ========================================
% 第3節 データの散らばりと四分位範囲
% ========================================

\section{データの散らばりと四分位範囲}

データの代表値だけでは, データの特徴を十分に表せないことがあります .

データがどれだけ散らばっているかを表す値として, 範囲, 四分位数, 四分位範囲などがあります .

\subsection{範囲(レンジ)}

\begin{definitionbox}{}{範囲(range)}
データの最大値と最小値の差 .

\[
\text{範囲} = \text{最大値} - \text{最小値}
\]

\vspace{0.5em}
\textbf{特徴:}
\begin{itemize}
\item 計算が簡単
\item 極端な値の影響を受けやすい
\item データの散らばり具合の大まかな目安になる
\end{itemize}
\end{definitionbox}

\vspace{1em}

\begin{example}[7]
次のデータの範囲を求めなさい .

\[
12, \, 18, \, 15, \, 22, \, 14, \, 20, \, 16, \, 19
\]

\begin{solution}
最大値は22, 最小値は12なので,

\[
\text{範囲} = 22 - 12 = 10
\]

よって, 範囲は \textcolor{blue}{\textbf{10}} である .
\end{solution}
\end{example}

\vspace{2em}

\subsection{四分位数}

データを4等分する位置の値を四分位数といいます .

\begin{definitionbox}{}{四分位数(quartile)}
データを小さい順に並べたとき, データを4等分する3つの値 .

\vspace{0.5em}
\begin{itemize}
\item \textbf{第1四分位数($Q_1$)}:下位25\%の位置の値
\item \textbf{第2四分位数($Q_2$)}:下位50\%の位置の値(中央値と同じ)
\item \textbf{第3四分位数($Q_3$)}:下位75\%の位置の値
\end{itemize}

\vspace{1em}
\begin{center}
\begin{tikzpicture}[scale=0.8]
% 数直線
\draw[thick,->] (0,0) -- (13,0) node[right] {データ};
% 四分位数の位置
\draw[thick] (2,0.2) -- (2,-0.2) node[below] {最小値};
\draw[thick,red] (4,0.2) -- (4,-0.2) node[below,red] {$Q_1$};
\draw[thick,blue] (6.5,0.2) -- (6.5,-0.2) node[below,blue] {$Q_2$};
\draw[thick,red] (9,0.2) -- (9,-0.2) node[below,red] {$Q_3$};
\draw[thick] (11,0.2) -- (11,-0.2) node[below] {最大値};
% 領域の表示
\node at (3,0.8) {25\%};
\node at (5.25,0.8) {25\%};
\node at (7.75,0.8) {25\%};
\node at (10,0.8) {25\%};
\end{tikzpicture}
\end{center}
\end{definitionbox}

\vspace{2em}

\subsection{四分位数の求め方}

\begin{tcolorbox}[colback=blue!5, colframe=blue!70!black, title=四分位数の求め方]
\textbf{手順:}
\begin{enumerate}
\item データを小さい順に並べる
\item 中央値($Q_2$)を求める
\item 中央値より小さいデータの中央値が $Q_1$
\item 中央値より大きいデータの中央値が $Q_3$
\end{enumerate}

\vspace{0.5em}
\textbf{注意:}
\begin{itemize}
\item データの個数が奇数の場合, 中央値は含めない
\item データの個数が偶数の場合, 半分に分けて考える
\end{itemize}
\end{tcolorbox}

\vspace{2em}

\begin{example}[8]
次のデータの四分位数を求めなさい .

\[
8, \, 12, \, 15, \, 10, \, 18, \, 14, \, 20, \, 11, \, 16, \, 13, \, 19
\]

\begin{solution}
\textbf{Step 1:} データを小さい順に並べる

\[
8, \, 10, \, 11, \, 12, \, 13, \, \textcolor{blue}{\textbf{14}}, \, 15, \, 16, \, 18, \, 19, \, 20
\]

\vspace{0.5em}
データは11個(奇数)なので, 6番目の値が中央値 .

\[
Q_2 = 14
\]

\vspace{1.5em}

\textbf{Step 2:} $Q_1$ を求める(中央値より小さいデータ)

\[
8, \, 10, \, \textcolor{red}{\textbf{11}}, \, 12, \, 13
\]

\vspace{0.5em}
5個(奇数)なので, 3番目の値が $Q_1$ .

\[
Q_1 = 11
\]

\vspace{1.5em}

\textbf{Step 3:} $Q_3$ を求める(中央値より大きいデータ)

\[
15, \, 16, \, \textcolor{red}{\textbf{18}}, \, 19, \, 20
\]

\vspace{0.5em}
5個(奇数)なので, 3番目の値が $Q_3$ .

\[
Q_3 = 18
\]

\vspace{1em}

よって, $Q_1 = \textcolor{blue}{\textbf{11}}$, $Q_2 = \textcolor{blue}{\textbf{14}}$, $Q_3 = \textcolor{blue}{\textbf{18}}$ である .
\end{solution}
\end{example}

\newpage

\begin{problem}[四分位数の計算]
次の各データの四分位数 $Q_1$, $Q_2$, $Q_3$ を求めなさい .

\begin{enumerate}
\item $5, \, 8, \, 12, \, 15, \, 18, \, 20, \, 22, \, 25, \, 28$
\vspace{8cm}

\item $10, \, 14, \, 16, \, 18, \, 20, \, 22, \, 24, \, 26, \, 28, \, 30$
\vspace{8cm}
\end{enumerate}
\end{problem}

\vspace{2em}

\subsection{四分位範囲と四分位偏差}

\begin{definitionbox}{}{四分位範囲(IQR)}
第3四分位数と第1四分位数の差 .

\[
\text{四分位範囲} = Q_3 - Q_1
\]

\vspace{0.5em}
データの中央50\%の散らばり具合を表す .

\vspace{0.5em}
\textbf{特徴:}
\begin{itemize}
\item 極端な値の影響を受けにくい
\item データの散らばり具合を安定的に表せる
\item 箱ひげ図で視覚的に表現できる
\end{itemize}
\end{definitionbox}

\vspace{1em}

\begin{definitionbox}{}{四分位偏差}
四分位範囲の半分の値 .

\[
\text{四分位偏差} = \, \frac{Q_3 - Q_1}{2} \,
\]

\vspace{0.5em}
データの散らばりを表す別の指標 .
\end{definitionbox}

\vspace{2em}

\begin{example}[9]
例題8のデータについて, 四分位範囲と四分位偏差を求めなさい .

\begin{solution}
例題8より, $Q_1 = 11$, $Q_3 = 18$ なので,

\begin{align*}
\text{四分位範囲} &= Q_3 - Q_1 \\[0.8em]
&= 18 - 11 \\[0.8em]
&= 7
\end{align*}

\vspace{1em}

\begin{align*}
\text{四分位偏差} &= \, \frac{Q_3 - Q_1}{2} \, \\[1em]
&= \, \frac{18 - 11}{2} \, \\[1em]
&= \, \frac{7}{2} \, \\[1em]
&= 3.5
\end{align*}

\vspace{0.5em}
よって, 四分位範囲は \textcolor{blue}{\textbf{7}}, 四分位偏差は \textcolor{blue}{\textbf{3.5}} である .
\end{solution}
\end{example}

\vspace{2em}

\subsection{箱ひげ図}

四分位数を使って, データの分布を視覚的に表したグラフを箱ひげ図といいます .

\begin{definitionbox}{}{箱ひげ図(box plot)}
データの最小値, $Q_1$, $Q_2$, $Q_3$, 最大値の5つの値を使って, データの分布を表すグラフ .

\vspace{1em}
\begin{center}
\begin{tikzpicture}[scale=0.6]
% 数直線
\draw[thick,->] (0,0) -- (14,0);
\foreach \x in {0,2,4,6,8,10,12}
    \node at (\x,-0.5) {\x};

% 箱ひげ図の例
\draw[thick] (2,1) -- (2,3);  % 最小値
\draw[thick] (2,2) -- (4,2);  % ひげ(左)
\draw[thick,fill=blue!20] (4,1) rectangle (10,3);  % 箱
\draw[thick,red] (7,1) -- (7,3);  % 中央値
\draw[thick] (10,2) -- (12,2);  % ひげ(右)
\draw[thick] (12,1) -- (12,3);  % 最大値

% ラベル
\node[below] at (2,0.5) {最小値};
\node[below] at (4,0.5) {$Q_1$};
\node[below,red] at (7,0.5) {$Q_2$};
\node[below] at (10,0.5) {$Q_3$};
\node[below] at (12,0.5) {最大値};

% 四分位範囲の表示
\draw[<->,thick,blue] (4,4) -- (10,4);
\node[above,blue] at (7,4) {四分位範囲};
\end{tikzpicture}
\end{center}

\vspace{0.5em}
\textbf{読み取れる情報:}
\begin{itemize}
\item データの中央値と散らばり
\item データの分布の偏り
\item 複数のデータの比較
\end{itemize}
\end{definitionbox}

\vspace{2em}

\begin{problem}[四分位範囲と箱ひげ図]
2つのクラスA, Bで数学のテストを行った . 結果は次の通りである .

\vspace{0.5em}
\textbf{クラスA:} 最小値50点, $Q_1$65点, $Q_2$72点, $Q_3$80点, 最大値95点

\vspace{0.5em}
\textbf{クラスB:} 最小値55点, $Q_1$62点, $Q_2$70点, $Q_3$78点, 最大値90点

\vspace{1em}
\begin{enumerate}
\item 各クラスの四分位範囲を求めなさい .

\vspace{5cm}

\item 各クラスの箱ひげ図を描き, 成績の分布を比較しなさい .

\vspace{10cm}
\end{enumerate}
\end{problem}

\vspace{2cm}

\subsection{まとめ:データの散らばりを表す指標}

\begin{tcolorbox}[colback=green!5, colframe=green!70!black, title=散らばりの指標のまとめ]
\begin{center}
\begin{tabular}{|c|c|p{6cm}|}
\hline
\textbf{指標} & \textbf{計算式} & \textbf{特徴} \\
\hline
範囲 & 最大値 $-$ 最小値 & 簡単だが極端な値の影響を受けやすい \\
\hline
四分位範囲 & $Q_3 - Q_1$ & 極端な値の影響を受けにくい \\
\hline
四分位偏差 & $\, \frac{Q_3 - Q_1}{2} \,$ & 四分位範囲の半分 \\
\hline
\end{tabular}
\end{center}
\end{tcolorbox}

\vspace{2em}

\begin{problem}[総合問題]
次のデータは, ある学校の生徒10人の読書時間(分/日)である .

\[
30, \, 45, \, 20, \, 60, \, 35, \, 50, \, 25, \, 55, \, 40, \, 70
\]

\begin{enumerate}
\item 平均値, 中央値を求めなさい .

\vspace{7cm}

\item 四分位数 $Q_1$, $Q_2$, $Q_3$ を求めなさい .

\vspace{7cm}

\item 範囲と四分位範囲を求めなさい .

\vspace{7cm}

\item 箱ひげ図を描きなさい .

\vspace{8cm}
\end{enumerate}
\end{problem}

\vspace{2cm}




\end{document}
