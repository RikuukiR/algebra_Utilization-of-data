% ========================================
% 第1節 データの整理
% ========================================

\section{データの整理}

\subsection{度数分布表とヒストグラム}

\begin{definitionbox}[def:度数分布表]{\textbf{度数分布表}}
\begin{minipage}[t]{0.55\textwidth}
    データの範囲を適当に区切ったとき、各区間に含まれるデータの個数を \textbf{度数} といい、
    各区間にその区間の度数を対応させて整理した右のような表を \textbf{度数分布表} という.

    \vspace{0.5em}

    \textbf{用語:}
    \begin{itemize}
        \item \textbf{階級}:データを分けた区間
        \item \textbf{階級値}:各階級の中央の値
        \item \textbf{度数}:各階級に含まれるデータの個数
    \end{itemize}
\end{minipage}%
\hfill
\begin{minipage}[t]{0.4\textwidth}
    \centering
    \vspace{0pt}
    \textbf{度数分布表}

    \vspace{0.3em}
    \begin{tabular}{|r|c|}
    \hline
    \textbf{階級(点)} & \textbf{度数(人)} \\
    \hline
    50〜60 & 2 \\
    \hline
    60〜70 & 6 \\
    \hline
    70〜80 & 9 \\
    \hline
    80〜90 & 3 \\
    \hline
    \textbf{計} & \textbf{20} \\
    \hline
    \end{tabular}
\end{minipage}
\end{definitionbox}

\begin{definitionbox}[def:階級]{\textbf{階級}}
    度数分布表において、区切られた各区間を \textbf{階級} 、区間の幅を \textbf{階級の幅} 、
    各階級の中央の値を \textbf{階級値} という.
\end{definitionbox}

\begin{definitionbox}[def:ヒストグラム]{\textbf{ヒストグラム}}
    度数分布表を、柱状グラフで
\end{definitionbox}

\begin{example}[1]{度数分布表の作成}
\begin{solution}
階級の幅を10点として整理すると、次の度数分布表が得られる。

\vspace{1em}
\begin{center}
\begin{tabular}{|c|c|c|}
\hline
\textbf{階級(点)} & \textbf{階級値(点)} & \textbf{度数(人)} \\
\hline
50以上〜60未満 & 55 & 2 \\
\hline
60以上〜70未満 & 65 & 6 \\
\hline
70以上〜80未満 & 75 & 9 \\
\hline
80以上〜90未満 & 85 & 3 \\
\hline
\textbf{合計} & - & \textbf{20} \\
\hline
\end{tabular}
\end{center}
\end{solution}
\end{example}

\vspace{2em}

\subsection{ヒストグラム}

度数分布表を視覚的に表現するために、ヒストグラムを使います。

\begin{definitionbox}{}{ヒストグラム}
度数分布表をもとに、横軸に階級、縦軸に度数をとり、各階級の度数を長方形の柱で表したグラフ。
\begin{itemize}
\item 長方形の面積が度数を表す
\item 階級の幅が等しい場合、柱の高さが度数を表す
\item データの分布の様子が視覚的にわかりやすい
\end{itemize}
\end{definitionbox}

\begin{problem}[度数分布表とヒストグラム]
次のデータは、ある中学校の生徒30人の通学時間(分)である。

\vspace{0.5em}
\begin{center}
15, 22, 8, 35, 18, 25, 12, 28, 20, 16, \\
30, 14, 26, 19, 23, 11, 32, 17, 24, 21, \\
27, 13, 29, 10, 31, 15, 20, 18, 25, 22
\end{center}
\vspace{0.5em}

\begin{enumerate}
\item 階級の幅を5分として度数分布表を作りなさい。
\item 度数分布表をもとに、ヒストグラムを作成しなさい。
\end{enumerate}
\end{problem}

\vspace{15cm}

\subsection{相対度数}

度数分布表では、各階級の度数がわかりますが、全体に対する割合を知りたいときには相対度数を使います。

\begin{definitionbox}{}{相対度数}
各階級の度数が、全体の度数に対してどれだけの割合を占めているかを表す値。

\[
\text{相対度数} = \frac{\text{その階級の度数}}{\text{全体の度数(総度数)}}
\]

\begin{itemize}
\item 相対度数の合計は常に1になる
\item パーセントで表すこともある(相対度数×100)
\end{itemize}
\end{definitionbox}

\begin{example}[2]
例題1の度数分布表に相対度数を加えなさい。

\begin{solution}
総度数は20人なので、各階級の相対度数を計算する。

\vspace{1em}
\begin{center}
\begin{tabular}{|c|c|c|c|}
\hline
\textbf{階級(点)} & \textbf{階級値} & \textbf{度数(人)} & \textbf{相対度数} \\
\hline
50以上〜60未満 & 55 & 2 & $\frac{2}{20} = 0.10$ \\
\hline
60以上〜70未満 & 65 & 6 & $\frac{6}{20} = 0.30$ \\
\hline
70以上〜80未満 & 75 & 9 & $\frac{9}{20} = 0.45$ \\
\hline
80以上〜90未満 & 85 & 3 & $\frac{3}{20} = 0.15$ \\
\hline
\textbf{合計} & - & \textbf{20} & \textbf{1.00} \\
\hline
\end{tabular}
\end{center}

\vspace{1em}
相対度数の合計は $0.10 + 0.30 + 0.45 + 0.15 = 1.00$ となる。
\end{solution}
\end{example}

\vspace{2em}

\begin{problem}[相対度数の計算]
次の度数分布表について、相対度数を求めなさい。

\vspace{1em}
\begin{center}
\begin{tabular}{|c|c|}
\hline
\textbf{階級(cm)} & \textbf{度数(人)} \\
\hline
140以上〜150未満 & 3 \\
\hline
150以上〜160未満 & 8 \\
\hline
160以上〜170未満 & 12 \\
\hline
170以上〜180未満 & 5 \\
\hline
180以上〜190未満 & 2 \\
\hline
\textbf{合計} & \textbf{30} \\
\hline
\end{tabular}
\end{center}
\end{problem}

\vspace{8cm}

\subsection{累積度数}

\begin{definitionbox}{}{累積度数}
ある階級までの度数の合計。データの分布の累積的な傾向を知ることができる。

\begin{itemize}
\item 最小の階級から順に度数を足していく
\item 最後の階級の累積度数は総度数と等しくなる
\item 累積相対度数も同様に計算できる
\end{itemize}
\end{definitionbox}

\vspace{1em}

\begin{problem}[累積度数]
例題1の度数分布表に累積度数を加えなさい。
\end{problem}

\vspace{8cm}

