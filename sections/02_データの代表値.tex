% ========================================
% 第2節 データの代表値
% ========================================

\section{データの代表値}

データの特徴を1つの値で表すとき、その値を\textbf{代表値}といいます。
代表値には、平均値、中央値(メジアン)、最頻値(モード)があります。

\subsection{平均値}

\begin{definitionbox}{}{平均値(mean)}
すべてのデータの値を足して、データの個数で割った値。

\[
\text{平均値} = \frac{\text{データの値の合計}}{\text{データの個数}}
\]

記号では、データを $x_1, x_2, \ldots, x_n$ とすると、平均値 $\mean{x}$ は次のように表される。

\[
\mean{x} = \frac{x_1 + x_2 + \cdots + x_n}{n} = \frac{1}{n}\sum_{i=1}^{n} x_i
\]

\textbf{特徴:}
\begin{itemize}
\item すべてのデータが平均値に関わる
\item 極端に大きい値や小さい値の影響を受けやすい
\end{itemize}
\end{definitionbox}

\vspace{1em}

\begin{example}[3]
次のデータの平均値を求めなさい。

\[
5, 8, 6, 9, 7, 10, 6, 8, 7, 9
\]

\begin{solution}
\begin{align*}
\mean{x} &= \frac{5 + 8 + 6 + 9 + 7 + 10 + 6 + 8 + 7 + 9}{10} \\
&= \frac{75}{10} \\
&= 7.5
\end{align*}

よって、平均値は \textcolor{blue}{\textbf{7.5}} である。
\end{solution}
\end{example}

\vspace{2em}

\begin{problem}[平均値の計算]
次の各データの平均値を求めなさい。

\begin{enumerate}
\item $3, 5, 7, 9, 11$
\vspace{3cm}

\item $12, 15, 18, 14, 16, 13, 17$
\vspace{3cm}

\item $20, 25, 22, 28, 30, 24, 26, 23$
\vspace{3cm}
\end{enumerate}
\end{problem}

\newpage

\subsection{度数分布表からの平均値}

度数分布表が与えられている場合、階級値を使って平均値を求めます。

\begin{definitionbox}{}{度数分布表からの平均値}
度数分布表において、各階級の階級値を $x_i$、度数を $f_i$ とすると、平均値は次のように求められる。

\[
\mean{x} = \frac{x_1 f_1 + x_2 f_2 + \cdots + x_n f_n}{f_1 + f_2 + \cdots + f_n} = \frac{\sum x_i f_i}{\sum f_i}
\]

\textbf{計算手順:}
\begin{enumerate}
\item 各階級の階級値と度数の積を求める
\item すべての積の合計を求める
\item 合計を総度数で割る
\end{enumerate}
\end{definitionbox}

\vspace{1em}

\begin{example}[4]
次の度数分布表から、平均値を求めなさい。

\vspace{1em}
\begin{center}
\begin{tabular}{|c|c|c|}
\hline
\textbf{階級(点)} & \textbf{階級値(点)} & \textbf{度数(人)} \\
\hline
50以上〜60未満 & 55 & 2 \\
\hline
60以上〜70未満 & 65 & 6 \\
\hline
70以上〜80未満 & 75 & 9 \\
\hline
80以上〜90未満 & 85 & 3 \\
\hline
\textbf{合計} & - & \textbf{20} \\
\hline
\end{tabular}
\end{center}

\begin{solution}
各階級値と度数の積を計算する。

\begin{align*}
\mean{x} &= \frac{55 \times 2 + 65 \times 6 + 75 \times 9 + 85 \times 3}{20} \\
&= \frac{110 + 390 + 675 + 255}{20} \\
&= \frac{1430}{20} \\
&= 71.5
\end{align*}

よって、平均値は \textcolor{blue}{\textbf{71.5点}} である。
\end{solution}
\end{example}

\vspace{2em}

\subsection{中央値(メジアン)}

\begin{definitionbox}{}{中央値(median)}
データを大きさの順に並べたとき、中央に位置する値。

\textbf{求め方:}
\begin{itemize}
\item データの個数が奇数のとき:真ん中の値
\item データの個数が偶数のとき:真ん中の2つの値の平均
\end{itemize}

\textbf{特徴:}
\begin{itemize}
\item データを2等分する位置の値
\item 極端な値の影響を受けにくい
\end{itemize}
\end{definitionbox}

\vspace{1em}

\begin{example}[5]
次のデータの中央値を求めなさい。

(1) $3, 7, 5, 9, 6, 8, 4$

(2) $12, 15, 10, 18, 14, 16$

\begin{solution}
(1) データを大きさの順に並べる:$3, 4, 5, \textcolor{red}{\textbf{6}}, 7, 8, 9$

データは7個(奇数)なので、4番目の値が中央値。

よって、中央値は \textcolor{blue}{\textbf{6}} である。

\vspace{1em}

(2) データを大きさの順に並べる:$10, 12, \textcolor{red}{\textbf{14}}, \textcolor{red}{\textbf{15}}, 16, 18$

データは6個(偶数)なので、3番目と4番目の値の平均が中央値。

\[
\text{中央値} = \frac{14 + 15}{2} = \frac{29}{2} = 14.5
\]

よって、中央値は \textcolor{blue}{\textbf{14.5}} である。
\end{solution}
\end{example}

\vspace{2em}

\begin{problem}[中央値の計算]
次の各データの中央値を求めなさい。

\begin{enumerate}
\item $8, 12, 15, 10, 14, 11, 13$
\vspace{4cm}

\item $20, 25, 18, 22, 24, 19, 21, 23$
\vspace{4cm}
\end{enumerate}
\end{problem}

\newpage

\subsection{最頻値(モード)}

\begin{definitionbox}{}{最頻値(mode)}
データの中で最も多く現れる値。度数分布表では、度数が最も大きい階級の階級値。

\textbf{特徴:}
\begin{itemize}
\item データの中で最も頻繁に現れる値
\item 複数存在する場合もある
\item 存在しない場合もある
\end{itemize}
\end{definitionbox}

\vspace{1em}

\begin{example}[6]
次のデータの最頻値を求めなさい。

\[
5, 7, 8, 5, 9, 6, 5, 8, 7, 10
\]

\begin{solution}
各値の度数を数える:

\begin{center}
\begin{tabular}{c|cccccc}
値 & 5 & 6 & 7 & 8 & 9 & 10 \\
\hline
度数 & \textcolor{red}{\textbf{3}} & 1 & 2 & 2 & 1 & 1
\end{tabular}
\end{center}

5が3回で最も多く現れる。

よって、最頻値は \textcolor{blue}{\textbf{5}} である。
\end{solution}
\end{example}

\vspace{2em}

\subsection{3つの代表値の比較}

\begin{tcolorbox}[colback=yellow!10, colframe=orange!70!black, title=代表値のまとめ]
\begin{center}
\begin{tabular}{|c|p{4cm}|p{5cm}|}
\hline
\textbf{代表値} & \textbf{特徴} & \textbf{使い分け} \\
\hline
平均値 & すべてのデータを使う & 全体的な傾向を知りたいとき \\
\hline
中央値 & 極端な値の影響を受けにくい & データに極端な値があるとき \\
\hline
最頻値 & 最も多く現れる値 & どの値が多いか知りたいとき \\
\hline
\end{tabular}
\end{center}
\end{tcolorbox}

\vspace{1em}

\begin{problem}[代表値の計算]
次のデータについて、平均値、中央値、最頻値を求めなさい。

\[
6, 8, 7, 9, 6, 10, 7, 8, 6, 9, 7, 8
\]
\end{problem}

\vspace{10cm}

