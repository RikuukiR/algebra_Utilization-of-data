% ========================================
% 第2節 データの代表値
% ========================================

\section{データの代表値}

\begin{definitionbox}[def:代表値]{\textbf{代表値}}
    いくつかの値が集まったデータがあるとき、そのデータ全体の特徴を表す数値を、
    そのデータの \textbf{代表値} という。
\end{definitionbox}
\vspace{1em}

代表値には、平均値、中央値、最頻値がある。

\subsection{平均値}

\begin{definitionbox}[def:平均値]{\textbf{平均値}}
    $n$ 個の値が集まったデータがあるとする。

    これら $n$ 個の値の合計を個数 $n$ でわった値を、このデータの \textbf{平均値} という。
\end{definitionbox}

\begin{theorembox}[thm:平均値]{\textbf{平均値}}
    平均値は次の式で求められる。

    \[
    \textbf{(平均値)} = \dfrac{\textbf{(データの値の合計)}}{\textbf{(データの個数)}}
    \]
\end{theorembox}

% 練習5
\begin{exercise}[【教p.120 練習5】]
    次のデータは、ジョギングを日課にしているAさんが最近5日間に行ったジョギングの時間である。
    このデータの平均値を求めよ。

    \[
    23 \qquad\quad 18 \qquad\quad 35 \qquad\quad 27 \qquad\quad 42 \qquad \text{(単位は 分)}
    \]
\end{exercise}

    \begin{proof}\mbox{}\\
        \begin{alignanswer*}
            \textbf{(平均値)} &= \dfrac{23 + 18 + 35 + 27 + 42}{5} \\
            &= \dfrac{145}{5} \\
            &= 29 \text{(分)}
        \end{alignanswer*}
    \end{proof}
\newpage

\begin{definitionbox}[def:中央値]{\textbf{中央値}}
    \uwave{データを大きさの順に並べたとき}、その中央にくる値を、\textbf{中央値} または \textbf{メジアン} という。

    ただし、データの個数が偶数のとき、中央に2つの値が並ぶから、その2つの値の平均値を中央値とする。
\end{definitionbox}

% 練習6
\begin{exercise}[【教p.120 練習6】]
    次のデータは、あるクラスの生徒10人の英語のテストの得点である。
    このデータの中央値を求めよ。

    \[
    75 \qquad 38 \qquad 49 \qquad 88 \qquad 61 \qquad 83 \qquad 44 \qquad 67 \qquad 58 \qquad 95 \qquad \text{(単位は 点)}
    \]
\end{exercise}

    \begin{proof}\mbox{}\\
        10人の英語のテストの得点を、低い順に並べると

        \begin{alignanswer*}
            38 \qquad 44 \qquad 49 \qquad 58 \qquad 61 \qquad 67 \qquad 75 \qquad 83 \qquad 88 \qquad 95
        \end{alignanswer*}

        5番目と6番目の得点の平均値が中央値であるから

        \begin{alignanswer*}
            \textbf{(中央値)} &= \dfrac{61 + 67}{2} \\
            &= 64 \text{(点)}
        \end{alignanswer*}
    \end{proof}

\begin{definitionbox}[def:最頻値]{\textbf{最頻値}}
    データにおいて、最も個数の多い値を、そのデータの \textbf{最頻値} または \textbf{モード} という。

    \uwave{データが度数分布表に整理されているときは}、\textbf{度数が最も大きい階級の階級値を最頻値}とする。
\end{definitionbox}

% 練習7
\begin{exercise}[【教p.121 練習7改訂】]
    表\ref{tab:height_distribution}において、身長の最頻値を求めよ。
    このデータの最頻値を求めよ。
\end{exercise}
\newpage

度数分布表を利用したデータの平均値を求める方法を考えよう。データが度数分布表にまとめられているとき、\textbf{ある階級に含まれるデータは、すべてその階級の階級値をとるものと考えて} 平均値を求める。

\begin{theorembox}[thm:度数分布表を利用した平均値]{\textbf{度数分布表を利用した平均値}}
    度数分布表を利用した平均値は次の式で求められる。
    \[
    \textbf{(平均値)} = \dfrac{\{\textbf{\textbf{(階級値)}} \bm{\times} \textbf{\textbf{(度数)}}\} \textbf{の合計}}
    {\textbf{\textbf{(度数の合計)}}}
    \]
\end{theorembox}

% 追加例題3
\begin{example}[【教p.121 追加例題3】]
    表\ref{tab:height_distribution}において, 次の問いに答えなさい.
    \begin{enumerate}
        \item 身長の平均値を求めなさい.
        \item 身長の中央値を求めなさい.
    \end{enumerate}
\end{example}

\begin{proof}\mbox{}\\
    \begin{table}[H]
        \centering
        \begin{tabular}{|r@{〜}l|c|c|c|}
        \hline
        \multicolumn{2}{|c|}{\textbf{階級(cm)}} & \textbf{度数(人)} & \textbf{階級値} & \textbf{階級値}$\bm{\times}$\textbf{度数} \\
        \hline
        135{\scriptsize 以上} & 140{\scriptsize 未満} & 2 & \answertable{137.5} & \answertable{275} \\
        \hline
        140 & 145 & 4 & \answertable{142.5} & \answertable{570} \\
        \hline
        145 & 150 & 5 & \answertable{147.5} & \answertable{737.5} \\
        \hline
        150 & 155 & 8 & \answertable{152.5} & \answertable{1220} \\
        \hline
        155 & 160 & 11 & \answertable{157.5} & \answertable{1732.5} \\
        \hline
        160 & 165 & 9 & \answertable{162.5} & \answertable{1462.5} \\
        \hline
        165 & 170 & 7 & \answertable{167.5} & \answertable{1172.5} \\
        \hline
        170 & 175 & 4 & \answertable{172.5} & \answertable{690} \\
        \hline
        \multicolumn{2}{|c|}{\textbf{計}} & \textbf{50} & & \answertable{7860} \\
        \hline
        \end{tabular}
        \label{tab:height_average}
    \end{table}

    \begin{alignanswer*}
        \text{(平均値)} = \dfrac{7860}{50} = \answermath{157.2} \text{(cm)}\\
        \text{(中央値)} = \answermath{155.5} \text{(cm)}
    \end{alignanswer*}
    \vspace{-6em}
\end{proof}

\newpage
