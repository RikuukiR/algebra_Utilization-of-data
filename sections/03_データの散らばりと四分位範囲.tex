% ========================================
% 第3節 データの散らばりと四分位範囲
% ========================================

\section{データの散らばりと四分位範囲}

データの代表値だけでは, データの特徴を十分に表せないことがあります .

データがどれだけ散らばっているかを表す値として, 範囲, 四分位数, 四分位範囲などがあります .

\subsection{範囲(レンジ)}

\begin{definitionbox}{}{範囲(range)}
データの最大値と最小値の差 .

\[
\text{範囲} = \text{最大値} - \text{最小値}
\]

\vspace{0.5em}
\textbf{特徴:}
\begin{itemize}
\item 計算が簡単
\item 極端な値の影響を受けやすい
\item データの散らばり具合の大まかな目安になる
\end{itemize}
\end{definitionbox}

\vspace{1em}

\begin{example}[7]
次のデータの範囲を求めなさい .

\[
12, \, 18, \, 15, \, 22, \, 14, \, 20, \, 16, \, 19
\]

\begin{solution}
最大値は22, 最小値は12なので,

\[
\text{範囲} = 22 - 12 = 10
\]

よって, 範囲は \textcolor{blue}{\textbf{10}} である .
\end{solution}
\end{example}

\vspace{2em}

\subsection{四分位数}

データを4等分する位置の値を四分位数といいます .

\begin{definitionbox}{}{四分位数(quartile)}
データを小さい順に並べたとき, データを4等分する3つの値 .

\vspace{0.5em}
\begin{itemize}
\item \textbf{第1四分位数($Q_1$)}:下位25\%の位置の値
\item \textbf{第2四分位数($Q_2$)}:下位50\%の位置の値(中央値と同じ)
\item \textbf{第3四分位数($Q_3$)}:下位75\%の位置の値
\end{itemize}

\vspace{1em}
\begin{center}
\begin{tikzpicture}[scale=0.8]
% 数直線
\draw[thick,->] (0,0) -- (13,0) node[right] {データ};
% 四分位数の位置
\draw[thick] (2,0.2) -- (2,-0.2) node[below] {最小値};
\draw[thick,red] (4,0.2) -- (4,-0.2) node[below,red] {$Q_1$};
\draw[thick,blue] (6.5,0.2) -- (6.5,-0.2) node[below,blue] {$Q_2$};
\draw[thick,red] (9,0.2) -- (9,-0.2) node[below,red] {$Q_3$};
\draw[thick] (11,0.2) -- (11,-0.2) node[below] {最大値};
% 領域の表示
\node at (3,0.8) {25\%};
\node at (5.25,0.8) {25\%};
\node at (7.75,0.8) {25\%};
\node at (10,0.8) {25\%};
\end{tikzpicture}
\end{center}
\end{definitionbox}

\vspace{2em}

\subsection{四分位数の求め方}

\begin{tcolorbox}[colback=blue!5, colframe=blue!70!black, title=四分位数の求め方]
\textbf{手順:}
\begin{enumerate}
\item データを小さい順に並べる
\item 中央値($Q_2$)を求める
\item 中央値より小さいデータの中央値が $Q_1$
\item 中央値より大きいデータの中央値が $Q_3$
\end{enumerate}

\vspace{0.5em}
\textbf{注意:}
\begin{itemize}
\item データの個数が奇数の場合, 中央値は含めない
\item データの個数が偶数の場合, 半分に分けて考える
\end{itemize}
\end{tcolorbox}

\vspace{2em}

\begin{example}[8]
次のデータの四分位数を求めなさい .

\[
8, \, 12, \, 15, \, 10, \, 18, \, 14, \, 20, \, 11, \, 16, \, 13, \, 19
\]

\begin{solution}
\textbf{Step 1:} データを小さい順に並べる

\[
8, \, 10, \, 11, \, 12, \, 13, \, \textcolor{blue}{\textbf{14}}, \, 15, \, 16, \, 18, \, 19, \, 20
\]

\vspace{0.5em}
データは11個(奇数)なので, 6番目の値が中央値 .

\[
Q_2 = 14
\]

\vspace{1.5em}

\textbf{Step 2:} $Q_1$ を求める(中央値より小さいデータ)

\[
8, \, 10, \, \textcolor{red}{\textbf{11}}, \, 12, \, 13
\]

\vspace{0.5em}
5個(奇数)なので, 3番目の値が $Q_1$ .

\[
Q_1 = 11
\]

\vspace{1.5em}

\textbf{Step 3:} $Q_3$ を求める(中央値より大きいデータ)

\[
15, \, 16, \, \textcolor{red}{\textbf{18}}, \, 19, \, 20
\]

\vspace{0.5em}
5個(奇数)なので, 3番目の値が $Q_3$ .

\[
Q_3 = 18
\]

\vspace{1em}

よって, $Q_1 = \textcolor{blue}{\textbf{11}}$, $Q_2 = \textcolor{blue}{\textbf{14}}$, $Q_3 = \textcolor{blue}{\textbf{18}}$ である .
\end{solution}
\end{example}

\newpage

\begin{problem}[四分位数の計算]
次の各データの四分位数 $Q_1$, $Q_2$, $Q_3$ を求めなさい .

\begin{enumerate}
\item $5, \, 8, \, 12, \, 15, \, 18, \, 20, \, 22, \, 25, \, 28$
\vspace{8cm}

\item $10, \, 14, \, 16, \, 18, \, 20, \, 22, \, 24, \, 26, \, 28, \, 30$
\vspace{8cm}
\end{enumerate}
\end{problem}

\vspace{2em}

\subsection{四分位範囲と四分位偏差}

\begin{definitionbox}{}{四分位範囲(IQR)}
第3四分位数と第1四分位数の差 .

\[
\text{四分位範囲} = Q_3 - Q_1
\]

\vspace{0.5em}
データの中央50\%の散らばり具合を表す .

\vspace{0.5em}
\textbf{特徴:}
\begin{itemize}
\item 極端な値の影響を受けにくい
\item データの散らばり具合を安定的に表せる
\item 箱ひげ図で視覚的に表現できる
\end{itemize}
\end{definitionbox}

\vspace{1em}

\begin{definitionbox}{}{四分位偏差}
四分位範囲の半分の値 .

\[
\text{四分位偏差} = \, \frac{Q_3 - Q_1}{2} \,
\]

\vspace{0.5em}
データの散らばりを表す別の指標 .
\end{definitionbox}

\vspace{2em}

\begin{example}[9]
例題8のデータについて, 四分位範囲と四分位偏差を求めなさい .

\begin{solution}
例題8より, $Q_1 = 11$, $Q_3 = 18$ なので,

\begin{align*}
\text{四分位範囲} &= Q_3 - Q_1 \\[0.8em]
&= 18 - 11 \\[0.8em]
&= 7
\end{align*}

\vspace{1em}

\begin{align*}
\text{四分位偏差} &= \, \frac{Q_3 - Q_1}{2} \, \\[1em]
&= \, \frac{18 - 11}{2} \, \\[1em]
&= \, \frac{7}{2} \, \\[1em]
&= 3.5
\end{align*}

\vspace{0.5em}
よって, 四分位範囲は \textcolor{blue}{\textbf{7}}, 四分位偏差は \textcolor{blue}{\textbf{3.5}} である .
\end{solution}
\end{example}

\vspace{2em}

\subsection{箱ひげ図}

四分位数を使って, データの分布を視覚的に表したグラフを箱ひげ図といいます .

\begin{definitionbox}{}{箱ひげ図(box plot)}
データの最小値, $Q_1$, $Q_2$, $Q_3$, 最大値の5つの値を使って, データの分布を表すグラフ .

\vspace{1em}
\begin{center}
\begin{tikzpicture}[scale=0.6]
% 数直線
\draw[thick,->] (0,0) -- (14,0);
\foreach \x in {0,2,4,6,8,10,12}
    \node at (\x,-0.5) {\x};

% 箱ひげ図の例
\draw[thick] (2,1) -- (2,3);  % 最小値
\draw[thick] (2,2) -- (4,2);  % ひげ(左)
\draw[thick,fill=blue!20] (4,1) rectangle (10,3);  % 箱
\draw[thick,red] (7,1) -- (7,3);  % 中央値
\draw[thick] (10,2) -- (12,2);  % ひげ(右)
\draw[thick] (12,1) -- (12,3);  % 最大値

% ラベル
\node[below] at (2,0.5) {最小値};
\node[below] at (4,0.5) {$Q_1$};
\node[below,red] at (7,0.5) {$Q_2$};
\node[below] at (10,0.5) {$Q_3$};
\node[below] at (12,0.5) {最大値};

% 四分位範囲の表示
\draw[<->,thick,blue] (4,4) -- (10,4);
\node[above,blue] at (7,4) {四分位範囲};
\end{tikzpicture}
\end{center}

\vspace{0.5em}
\textbf{読み取れる情報:}
\begin{itemize}
\item データの中央値と散らばり
\item データの分布の偏り
\item 複数のデータの比較
\end{itemize}
\end{definitionbox}

\vspace{2em}

\begin{problem}[四分位範囲と箱ひげ図]
2つのクラスA, Bで数学のテストを行った . 結果は次の通りである .

\vspace{0.5em}
\textbf{クラスA:} 最小値50点, $Q_1$65点, $Q_2$72点, $Q_3$80点, 最大値95点

\vspace{0.5em}
\textbf{クラスB:} 最小値55点, $Q_1$62点, $Q_2$70点, $Q_3$78点, 最大値90点

\vspace{1em}
\begin{enumerate}
\item 各クラスの四分位範囲を求めなさい .

\vspace{5cm}

\item 各クラスの箱ひげ図を描き, 成績の分布を比較しなさい .

\vspace{10cm}
\end{enumerate}
\end{problem}

\vspace{2cm}

\subsection{まとめ:データの散らばりを表す指標}

\begin{tcolorbox}[colback=green!5, colframe=green!70!black, title=散らばりの指標のまとめ]
\begin{center}
\begin{tabular}{|c|c|p{6cm}|}
\hline
\textbf{指標} & \textbf{計算式} & \textbf{特徴} \\
\hline
範囲 & 最大値 $-$ 最小値 & 簡単だが極端な値の影響を受けやすい \\
\hline
四分位範囲 & $Q_3 - Q_1$ & 極端な値の影響を受けにくい \\
\hline
四分位偏差 & $\, \frac{Q_3 - Q_1}{2} \,$ & 四分位範囲の半分 \\
\hline
\end{tabular}
\end{center}
\end{tcolorbox}

\vspace{2em}

\begin{problem}[総合問題]
次のデータは, ある学校の生徒10人の読書時間(分/日)である .

\[
30, \, 45, \, 20, \, 60, \, 35, \, 50, \, 25, \, 55, \, 40, \, 70
\]

\begin{enumerate}
\item 平均値, 中央値を求めなさい .

\vspace{7cm}

\item 四分位数 $Q_1$, $Q_2$, $Q_3$ を求めなさい .

\vspace{7cm}

\item 範囲と四分位範囲を求めなさい .

\vspace{7cm}

\item 箱ひげ図を描きなさい .

\vspace{8cm}
\end{enumerate}
\end{problem}

\vspace{2cm}


